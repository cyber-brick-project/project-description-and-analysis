\section{Introduction}

This document covers project scope, devices to be build and history of creation. Technical analysis
of used part are covered in separate document. Technical details of each device are covered in
dedicated documents.

\subsection{Project scope}

Project is dedicated to build automated Lego rail network. System should serve travelers and
deliver cargo. Trains should be loaded and unloaded automatically.

Side projects can appear from time to time, and can be unrelated to main purpose.\\
Why? Just for fun!

%% ------------------------------------------------------------------------------------------ %%

\subsection{How it all started}

Cyber Brick Project was stared out of dreams of young boy. Dreams of better, bigger and more
complex creations, buildings, machines. Finally, whole cities, with working infrastructure.

These dreams were getting stronger after visiting Lego exhibitions, where such cities exist. But,
while such masterpieces were amazing, most of them had no life in them. Sometimes simple electric
engines were giving constant or repetitive motions. Like simulation of life.

Meanwhile, discovering models with remote controllers, gave hope, yet required person to
operate it all manually. And suddenly, there was pure revelation: ``AlmightyArjen''\footnote{
More details can be found at: \Uurl{http://www.almightyarjen.com/}}. Especially ``HUGE Lego train
coal terminal fully automated by Arduino''\footnote{ Youtubue video:
\Uurl{https://youtu.be/MEekXRNztVI}, description: \Uurl{http://www.almightyarjen.com/?page_id=80}}

While admiring AlmightyArjen's works, one common thing can be spotted. Most (if not all) trains
are powered from rails.

%% ------------------------------------------------------------------------------------------ %%
