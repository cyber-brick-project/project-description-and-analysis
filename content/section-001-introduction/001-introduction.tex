\section{Introduction}

This document covers project scope, devices to be build and project origins. Technical analysis
of used parts are covered in separate document. Technical details for each device is covered in
dedicated document.

%% ------------------------------------------------------------------------------------------ %%

\subsection{Project scope}

Project is dedicated to build automated Lego rail network. System should serve travelers and
deliver cargo. Trains should be loaded and unloaded automatically.

Side projects can appear from time to time, and can be unrelated to main purpose.\\
Why? Just for fun!

Side projects are covered in separate chapter.

%% ------------------------------------------------------------------------------------------ %%

\subsection{How it all started}

Cyber Brick Project was stared out of dreams of young boy. Dreams of better, bigger and more
complex creations, buildings, machines. Finally, whole cities, with working infrastructure.

These dreams were getting stronger after visiting Lego exhibitions, where such cities exist. But,
while such masterpieces were amazing, most of them had no life in them. Sometimes simple electric
engines were giving constant or repetitive motions. Like simulation of life.

Meanwhile, discovering models with remote controllers, gave hope, yet required person to
operate it all manually. And suddenly, there was pure revelation: ``AlmightyArjen''\footnote{
More details can be found at: \Uurl{http://www.almightyarjen.com/}}. Especially ``HUGE Lego train
coal terminal fully automated by Arduino''\footnote{ Youtubue video:
\Uurl{https://youtu.be/MEekXRNztVI}, description: \Uurl{http://www.almightyarjen.com/?page_id=80}}

%% ------------------------------------------------------------------------------------------ %%

\subsection{What can be achieved?}

While admiring AlmightyArjen's works, one common thing can be spotted. Most (if not all) trains
are powered from rails. Rails are splitted into segments, and all trains, which are on the same
segment, performs the same action.

Trains powered from rails are retired models, and not currently sold by Lego. Therefore it's
extremely hard to buy them.

Newer models has battery packs in them, and can be controlled separately. What if
microcontroller could be placed inside the train? Will it open new possibilities?

Cyber Brick Project focuses on building remotely controlled trains and whole infrastructure
for automated trains control, cargo processing and configurable schedule.

%% ------------------------------------------------------------------------------------------ %%
